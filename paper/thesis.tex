\documentclass{article}

\usepackage{fontspec}
\usepackage[backend=biber]{biblatex}
\usepackage[hidelinks]{hyperref}
\usepackage{mathtools,amssymb,stmaryrd}
\usepackage{tikz}
\usepackage{mathpartir}
\usepackage{tabularx}
\usepackage{booktabs}
\usepackage{listings}

\bibliography{sources.bib}

\newtheorem{theorem}{Theorem}

\newcommand{\lock}{
  \text{\tikz[baseline]{
      \fill[rounded corners=.1ex] (-.75ex,0) rectangle (.75ex,1ex);
      \draw[line width=.3ex] (-.4ex,.5ex) -- ++(0,.75ex) arc (180:0:.4ex) -- ++(0,-.75ex);
}}}

\DeclareMathOperator\Rpl{Rpl}
\DeclareMathOperator\unbox{unbox}

\begin{document}

\begin{titlepage}\centering

{\scshape\LARGE Master Thesis Half-Time Report\\}

\vspace{0.5cm}

{\huge\bfseries A Parametric Fitch-Style Modal Lambda Calculus\\}

\vspace{2cm}

{\Large Axel Forsman \texttt{<foraxel@student.chalmers.se>} \\}

\vspace{1.0cm}

{\large Supervisor: Nachiappan Valliappan \\}

\vspace{1.0cm}

{\large Examiner: Andreas Abel \\}

\vspace{1.5cm}

{\large Relevant completed courses:\par}

{\itshape \begin{itemize}
  \item DAT060, Logic in Computer Science; and
  \item DAT350, Types for Programs and Proofs.
  \end{itemize}}

\vfill
{\large April 17, 2023 \\}
\end{titlepage}

\section{Introduction}

The \emph{necessity modality}, denoted by $\Box$, where the focus will lie,
has been applied to model confidentiality in information-flow control,
compartmental purity in functional languages,
and more.
For a formula $\phi$, the formula $\Box \phi$ reads
``It is necessarily true that $\phi$'' -
$\Box$ changes the \emph{mode} of $\phi$.
So called Fitch-style modal deduction,
where modalities are eliminated by opening a subproof,
and introduced by shutting one,
has been adapted for lambda calculi.
Different modal logics may be encoded via different open and shut rules.
Prior work \cite{valliappan22} has given normalization proofs
for four Fitch-style formulations of lambda calculi with different modalities,
which required repeating the proofs for each individual calculus.
This prompts the need for a parametric Fitch-style modal lambda calculus
generalizing the variants,
in order to avoid repetition and ease further extensions.

\section{Background}

\textcite{fitch52} introduced a method for propositional deduction,
central to which is the idea of \emph{subordinate proof}.
For example, in order to prove $A \rightarrow B$ a subordinate proof may be opened
containing the new assumption $A$, where one sets out to prove $B$.
If successful, the subordinate proof can be shut
by introducing $A \rightarrow B$ in the original proof
thereby discharging the assumption $A$.

This has been adapted for modal lambda calculi \cite{borghuis94},
where modalities are type constructors that add some properties.
These may be understood using Kripke's possible worlds interpretation \cite{kripke63, huth04},
where opening a subproof means visiting a replica new world,
and shutting means returning.
To keep track of subordinate proofs in typing judgements
a new structural connective $\lock$ is added to the context when a $\Box$ is eliminated,
and popped when the subproof is closed.

\emph{Normalization by Evaluation} (NbE) is a technique for reducing terms to their normal forms,
which are not further reducible \cite{berger91}.
Instead of implementing the normalization procedure ``by hand'',
you instead proceed by evaluating,
before \emph{reifying} the resulting semantic value back into a term.
If at any point computation is blocked on a value known only at run time,
e.g. on an argument when descending into a lambda,
evaluation proceeds with a so called \emph{neutral value},
containing enough information about its origins to make reification possible.

Here we consider the family of Fitch-style modal lambda calculi
derived from intuitionistic propositional logic
extended with the unary connective $\Box$,
the inference rule \emph{necessitation}, if $\cdot \vdash A$ then $\Gamma \cdot \Box A$,
and different axioms surrounding $\Box$ \cite{clouston18}.
The most basic modal calculi, $\lambda_\text{IK}$,
comes from the \emph{K axiom}
($\Box(A \rightarrow B) \rightarrow \Box A \rightarrow \Box B$).
K together with \emph{axiom T} ($\Box A \rightarrow A$) gives $\lambda_\text{IT}$;
K and \emph{axiom 4} ($\Box A \rightarrow \Box\Box A$) yield $\lambda_\text{IK4}$;
and K, T and 4 give $\lambda_\text{IS4}$.

\textcite{valliappan22} noted that for the four lambda calculi
$\lambda_\text{IK}$, $\lambda_\text{IT}$, $\lambda_\text{IK4}$, $\lambda_\text{IS4}$,
only the $\Box$-elimination rules differ,
see figure~\ref{fig:elim-rules},
and chose to instead define the $\Box$-elimination rules
in terms of different \emph{modal accessibility relations} $\Delta\lhd\Gamma$.
The relation $\Delta\lhd\Gamma$ should be thought as saying
``the contents of boxed values from context $\Delta$ may be accessed in the future context $\Gamma$,''
as will become apparent upon seeing the generalized $\Box$-elimination rule.
In this paper,
we use that concept to formulate a single parametric calculus generalizing these four calculi.

\begin{figure}
  \begin{align*}
    &\inferrule[$\lambda_\text{IK}$/\Box-Elim]
    {\Gamma \vdash t : \Box A}
    {\Gamma, \Gamma' \vdash \unbox_{\lambda_\text{IK}} t : A}
    \lock \notin \Gamma' &
    &\inferrule[$\lambda_\text{IT}$/\Box-Elim]
          {\Gamma \vdash t : \Box A}
          {\Gamma, \Gamma' \vdash \unbox_{\lambda_\text{IT}} t : A}
          \#_{\lock} (\Gamma') \le 1 \\
          & \inferrule[$\lambda_\text{IK4}$/\Box-Elim]
            {\Gamma \vdash t : \Box A}
            {\Gamma, \lock, \Gamma' \vdash \unbox_{\lambda_\text{IK4}} t : A} &
            & \inferrule[$\lambda_\text{IS4}$/\Box-Elim]
            {\Gamma \vdash t : \Box A}
            {\Gamma, \Gamma' \vdash \unbox_{\lambda_\text{IS4}} t : A}
  \end{align*}
  \caption{$\Box$-elimination rules for the modal lambda calculi
    $\lambda_\text{IK}$, $\lambda_\text{IT}$, $\lambda_\text{IK4}$ and $\lambda_\text{IS4}$
    \cite{clouston18}.
    \label{fig:elim-rules}}
\end{figure}

The results are being formalized\footnote{Available online at
\url{https://github.com/axelf4/pfm-lambda}.}
in the proof assistant Agda \cite{norell07}.

\section{The calculus $\lambda_\text{PFM}$}

In this section we give the specification of the simply typed modal lambda calculus $\lambda_\text{PFM}$.
The calculus is parameterized by the binary relation $\Delta\lhd\Gamma$ on contexts,
subject to requirements that will be given below.

Types are constructed out of an uninterpreted base type $\iota$:
$$ \text{\emph{Type}} \quad A, B \Coloneqq \iota \mid A \rightarrow B \mid \Box A $$
Contexts are lists of types and locks:
$$ \text{\emph{Context}} \quad \Gamma \Coloneqq \cdot \mid \Gamma, A \mid \Gamma, \lock $$
The intrinsically typed syntax of the language is given in figure~\ref{fig:typing-rules},
where $\Gamma \vdash t : A$ is notation for $t$ being a well-typed term of type $T$ in context $\Gamma$.

\begin{figure}
  \centering
  \begin{align*}
    &\inferrule[Var]{ }{\Gamma, x : A, \Gamma' \vdash x : A} \, \lock \notin \Gamma' &
    &\inferrule[\rightarrow-Intro]{\Gamma, A \vdash t : B}{\Gamma \vdash \lambda t : A \rightarrow B} &
    &\inferrule[\rightarrow-Elim]{\Gamma \vdash t : A \rightarrow B \\ \Gamma \vdash s : A}{\Gamma \vdash t \; s : B}
  \end{align*}
  \begin{align*}
    &\inferrule[\Box-Intro]{\Gamma, \lock \vdash t : A}{\Gamma \vdash \operatorname{box} t : \Box A} &
    &\inferrule[\Box-Elim]{\Delta \vdash t : A \\ \Delta \lhd \Gamma}{\Gamma \vdash \operatorname{unbox} t : A}
  \end{align*}
  \caption{The set of intrinsically typed terms of $\lambda_\text{PFM}$.
    The modal accessibility relation $\Delta\lhd\Gamma$ is a parameter of the calculus.
    \label{fig:typing-rules}}
\end{figure}

In order to present the equational theory
we define OPE:s and substitutions.
An \emph{order-preserving embedding} (OPE) is a binary relation on contexts $\Gamma \subseteq \Delta$
signifying $\Gamma$ can be weakened, i.e. add more assumptions,
to obtain $\Delta$.
It is defined inductively as
\begin{equation*}
  \inferrule{ }{\operatorname{base} : \cdot \subseteq \cdot} \quad
  \inferrule{\Gamma \subseteq \Delta}{\operatorname{weak} : \Gamma \subseteq \Delta, A} \quad
  \inferrule{\Gamma \subseteq \Delta}{\operatorname{lift} : \Gamma, A \subseteq \Delta, A} \quad
  \inferrule{\Gamma \subseteq \Delta}{\operatorname{lift_\lock} : \Gamma, \lock \subseteq \Delta, \lock}
\end{equation*}
We define a map operation
$\operatorname{wk} : \Gamma\subseteq\Delta \rightarrow \Gamma \vdash t : A \rightarrow \Delta \vdash t : A$
that given an OPE weakens a term.
Only the case of weakening an $\operatorname{unbox}$ term is unlike the simply typed lambda calculus (STLC) counterpart:
\begin{lstlisting}[escapechar=\%,upquote=true]
  wk w (unbox t m) = let m' , w' = rewind%$_\subseteq$% m w
    in unbox (wk w' t) m'
\end{lstlisting}
where we require the calculus parameter
$$ \textit{rewind}_\subseteq : (m : \Gamma'\lhd\Gamma) \rightarrow (w : \Gamma\subseteq\Delta) \rightarrow \exists \Delta'. \, \Delta'\lhd\Delta \times \Gamma'\subseteq\Delta' $$
that given a modal accessibility relation $m$
truncates the contexts $\Gamma$ and $\Delta$ in $w$,
in order to remove as many locks from both as there are in $m$.
That is to say, it transports $w$ from the future world $\Gamma$ to instead act on the past world $\Gamma'$.

For substitutions - and later environments - we note that both
can be seen as replacement lists of items for each type in a context.
Thus we choose to define them as concrete instances of a type $\Rpl$,
parametric over some function
$F : \textit{Type} \rightarrow \textit{Context} \rightarrow \textbf{Set}$,
defined inductively as
\begin{equation*}
  \inferrule{ }{\cdot : \Rpl \cdot \; \Delta} \quad
  \inferrule{\sigma : \Rpl \Gamma \; \Delta \\ x : F \; A \; \Delta}
            {\sigma, x : \Rpl \; (\Gamma, A) \; \Delta} \quad
  \inferrule{\sigma : \Rpl \Gamma \; \Delta \\ m : \Delta\lhd\Delta'}
            {\operatorname{lock} \sigma \; m : \Rpl \; (\Gamma, \lock) \; \Delta'}
\end{equation*}
This helps unify some of the calculus parameters,
and avoids having the parameters depend on e.g. terms which in turn depend on other parameters.
Substitutions $\operatorname{Sub}$ may then be defined as $\Rpl$ with $F \; A \; \Gamma = \Gamma \vdash A$.

With the exception of the $\operatorname{lock}$ constructor
the definition of $\Rpl$ is as for substitutions in STLC.
Adding the alternate constructor
$\operatorname{lift}_\lock : \Rpl \Gamma \; \Delta \rightarrow \Rpl \; (\Gamma, \lock) \; (\Delta, \lock)$
it would be able to represent local substitutions in any of the ``worlds'' delimited by locks in the context.
Instead, $\operatorname{lock}$ with an argument $m : \Delta\lhd\Delta'$
(as used in \cite{valliappan22})
makes it possible to unify substitutions and the necessary \emph{modal transformations},
where locks are removed and added from contexts as permitted by $(\lhd)$.

With this choice of $\operatorname{lock}$ we make use of the parameter
$$ \lhd_\lock : \forall\Gamma. \, \Gamma \lhd \Gamma, \lock $$
in order to be able to define the identity substitution
$\operatorname{id}_s : \forall\Gamma. \, \operatorname{Sub} \Gamma \; \Gamma $

As for OPE:s we define
$\textit{subst} : \operatorname{Rpl} \Gamma \; \Delta \rightarrow \Gamma \vdash t : A \rightarrow \Delta \vdash t : A$,
using the parameter
$$ \textit{rewind} : (m : \Gamma'\lhd\Gamma) \rightarrow (\sigma : \operatorname{Rpl} \Gamma \; \Delta) \rightarrow \exists \Delta'. \, \Delta'\lhd\Delta \times \operatorname{Rpl} \Gamma' \; \Delta' $$

The equational theory of $\lambda_\text{PFM}$ is given in figure~\ref{fig:eq-theory},
where reflexivity, symmetry, transitivity and congruence rules have been omitted.
The notation $\Gamma \vdash t \sim s$ says that the terms $t$ and $s$
are equal in the context $\Gamma$ up to the conversion relation $(\sim)$.

\begin{figure}
  \centering
  \begin{align*}
    \text{$\beta$ equivalence:}& \quad
    \inferrule{\Gamma, A \vdash t : B \\ \Gamma \vdash s : A}{\Gamma \vdash (\lambda. \, t) \; s \sim t[s]} \quad
    \inferrule{\Delta, \lock \vdash t : A \\ m : \Delta \lhd \Gamma}{\Gamma \vdash \unbox \; (\operatorname{box} t) \; m \sim \textit{subst} \; (\operatorname{lock} \operatorname{id}_s m) \; t} \\
    \text{$\eta$ equivalence:}& \quad
    \inferrule{\Gamma \vdash t : A \rightarrow B}
         {\Gamma \vdash t \sim \lambda. \, (\operatorname{wk} \; (\operatorname{weak} \operatorname{id}_\subseteq) \; t) \; (\operatorname{var} \operatorname{zero})} \quad
         \inferrule{\Gamma \vdash t : \Box A}{\Gamma \vdash t \sim \operatorname{box} \; (\operatorname{unbox} t \; \lhd_\lock)}
  \end{align*}
  \caption{Equational theory of $\lambda_\text{PFM}$.
    The rules for lambda abstraction are as for STLC.
    In the $\lambda$-$\beta$-conversion rule,
    $t[s]$ denotes applying a singleton substitution to $t$:
    Replacing the zeroth variable with $s$
    and decrementing all other de Bruijn indices.
    \label{fig:eq-theory}}
\end{figure}

\section{Normalization algorithm}

We provide a NbE algorithm based on a possible worlds model.
Normal and neutral forms are defined mutually as:
\begin{gather*}
  \inferrule{x : \text{Ne} \; \Gamma \; \iota}{\operatorname{ne} x : \text{Nf} \; \Gamma \; \iota} \quad
  \inferrule{x : \text{Nf} \; (\Gamma, A) \; B}{\operatorname{abs} x : \text{Nf} \; \Gamma \; (A \rightarrow B)} \quad
  \inferrule{x : \text{Nf} \; (\Gamma, \lock) \; A}{\operatorname{box} x : \text{Nf} \; \Gamma \; (\Box A)} \\
  \inferrule{x : A \in \Gamma}{\operatorname{var} x : \text{Ne} \; \Gamma \; A} \quad
  \inferrule{x : \text{Ne} \; \Gamma \; (A \rightarrow B) \\ y : \text{Nf} \; \Gamma \; A}
            {x \; y : \text{Ne} \; \Gamma \; B} \quad
  \inferrule{x : \text{Ne} \; \Gamma \; (\Box A) \\ m : \Gamma'\lhd\Gamma}{\operatorname{unbox} x \; m : \text{Ne} \; \Gamma \; A}
\end{gather*}
The normal forms are $\eta$-long, i.e. all variables are maximally applied and unboxed,
as the $\operatorname{nf}$ constructor only permits neutral values of the base type.

As done in \cite{valliappan22}, we choose contexts for worlds,
weakenings for the intuitionistic accessibility relation between worlds, and
$(\lhd)$ for the modal accessibility relation.
(The intuitionistic accessibility relation should be thought of as
relating two worlds $w_1$ and $w_2$ if $w_2$ has as much or more knowledge than $w_1$;
for worlds as contexts this means all assumptions in $w_1$ should be present in $w_2$ too.)
Then we interpret types in the model as
\begin{align*}
  &\llbracket \iota \rrbracket_\Gamma \coloneqq \text{Nf} \; \Gamma \; \iota \\
  &\llbracket A \rightarrow B \rrbracket_\Gamma \coloneqq \forall \Delta. \, \Gamma \subseteq \Delta \rightarrow \llbracket A \rrbracket_\Delta \rightarrow \llbracket B \rrbracket_\Delta \\
  &\llbracket \Box A \rrbracket_\Gamma \coloneqq \forall \Gamma', \Delta. \, \Gamma \subseteq \Gamma' \rightarrow \Gamma'\lhd\Delta \rightarrow \llbracket A \rrbracket_\Delta
\end{align*}
and contexts as environments, i.e. lists of semantic values, using $\operatorname{Rpl}$,
$$ \llbracket \Gamma \rrbracket_\Delta \coloneqq \operatorname{Rpl}_{\llbracket\_\rrbracket} \Gamma \; \Delta $$
We have monotonicity for semantic values and environments,
i.e. we have
$wk_A : \Delta \subseteq \Delta' \rightarrow \llbracket A \rrbracket_\Delta \rightarrow \llbracket A \rrbracket_\Delta'$ and
$wk_\Gamma : \Delta \subseteq \Delta' \rightarrow \llbracket \Gamma \rrbracket_\Delta \rightarrow \llbracket \Gamma \rrbracket_\Delta'$.

The definition of evaluation, reification and reflection is given in figure~\ref{fig:nbe}.
The normalization function may then be given as
\begin{align*}
  &\textit{nf} : \Gamma \vdash t : A \rightarrow \operatorname{Nf} \Gamma \; A \\
  &\textit{nf} \; t \coloneqq \, \downarrow^A (\llbracket t \rrbracket \operatorname{id}_e)
\end{align*}
where $\operatorname{id}_e$ is the identity environment.
The algorithm can be summarized as follows:
Evaluation proceeds as for an interpreter,
except closures take an extra OPE argument;
this allows conjuring fresh variables under binders
when reifying functions.
Then the resulting semantic value is reified back to its normal form.
When reifying a box or function,
evaluation proceeds with the boxed term,
or the function applied to a neutral form corresponding to the argument type,
respectively.

\begin{figure}
  \centering
  \begin{align*}
    &\mathrlap{\text{Evaluation} \quad \llbracket\_\rrbracket : \Gamma \vdash t : A \rightarrow \forall\Delta. \, \llbracket\Gamma\rrbracket_\Delta \rightarrow \llbracket A \rrbracket_\Delta} &&\\
    &\llbracket x \rrbracket \; \hat\Gamma &&\coloneqq \text{lookup } x \text{ in } \hat\Gamma \\
    &\llbracket \lambda. \, t \rrbracket \; \hat\Gamma \; w \; \hat a &&\coloneqq \llbracket t \rrbracket \; (\operatorname{wk}_{\hat\Gamma} w \; \hat\Gamma, \hat a) \\
    &\llbracket t \; s \rrbracket \; \hat\Gamma &&\coloneqq \llbracket t \rrbracket \; \operatorname{id}_\subseteq \; (\llbracket s \rrbracket_{\hat\Gamma}) \\
    &\llbracket \operatorname{box} t \rrbracket \; \hat\Gamma \; w \; m &&\coloneqq \llbracket t \rrbracket \; (\operatorname{lock} \; (\operatorname{wk}_{\hat\Gamma} w \; \hat\Gamma) \; m) \\
    &\llbracket \operatorname{unbox} t \; m \rrbracket \; \hat\Gamma &&\coloneqq \llbracket t \rrbracket \; \hat\Delta \; \operatorname{id}_\subseteq m' \quad \text{where } m' , \hat\Delta = \textit{rewind} \; m \; \hat\Gamma \\
    &\mathrlap{\text{Reification} \quad \downarrow^A : \llbracket A \rrbracket_\Gamma \rightarrow \operatorname{Nf} \Gamma \; A} &&\\
    &\downarrow^\iota a &&\coloneqq a \\
    &\downarrow^{A \rightarrow B} a &&\coloneqq \operatorname{abs} \; (\downarrow^B \; (a \; (\operatorname{weak} \operatorname{id}_\subseteq) \; (\uparrow^A (\operatorname{var} \operatorname{zero})))) \\
    &\downarrow^{\Box A} a &&\coloneqq \operatorname{box} \; (\downarrow^A \; (a \; \operatorname{id}_\subseteq \lhd_\lock)) \\
    &\mathrlap{\text{Reflection} \quad \uparrow^A : \operatorname{Ne} \Gamma \; A \rightarrow \llbracket A \rrbracket_\Gamma} &&\\
    &\uparrow^\iota x &&\coloneqq \operatorname{nt} a \\
    &\uparrow^{A \rightarrow B} x \; w \; a &&\coloneqq \; \uparrow^B ((\operatorname{wk}_{\operatorname{Ne}} w \; x) \; \downarrow^A a) \\
    &\uparrow^{\Box A} x \; a &&\coloneqq \; \uparrow^B (\operatorname{unbox} \; (\operatorname{wk}_{\operatorname{Ne}} w \; x) \; m)
  \end{align*}
  \caption{Evaluation, reification and reflection definitions. \label{fig:nbe}}
\end{figure}

Here we have chosen the possible worlds inspired interpretation of $\Box A$,
instead of the syntax-directed approach of
$$ \llbracket \Box A \rrbracket_\Gamma = \llbracket A \rrbracket_{\Gamma, \lock}$$
one reason being that the $\unbox$ case of evaluation then would require
being able to apply the equivalent of a $\operatorname{lock} \operatorname{id} m$ substitution
on semantic values, or similar, in addition to weakening,
whereas currently no such thing is needed,
as the $m$ instead goes directly in $\unbox$ when reflecting.

\section{Completeness of the conversion relation}

Completeness of the conversion relation has been proved with respect to possible worlds,
with the addition of the following calculus parameters
\begin{itemize}
\item Rewinding $\operatorname{lock} \sigma \; m$
  with a modal accessibility relation $\Gamma \lhd \Gamma, \lock$
  should work as expected, i.e. give back $m$ and $\sigma$:
  $$ \textit{rewind-$\lhd_\lock$} : \forall m, \sigma. \, \textit{rewind} \lhd_\lock (\operatorname{lock} \sigma \; m) \equiv m , \sigma $$
  and the same for $\textit{rewind}_\subseteq$ on $\operatorname{lift_\lock}$.
\item The operation $\textit{rewind}$ should preserve composition:
  \begin{align*}
    \textit{rewindPres-$\circ$} &: (m : \Delta\lhd\Gamma) \rightarrow (\sigma_1 : \operatorname{Rpl} \Gamma \; \Gamma') \rightarrow (\sigma_2 : \operatorname{Rpl} \Gamma' \; \Gamma'') \\
    &\rightarrow
    \begin{array}[t]{@{}l@{}l@{}}
      \text{let } & m' , \sigma_1' = \textit{rewind} \; m \; \sigma_1 \\
      & m'' , \sigma_2' = \textit{rewind} \; m' \; \sigma_2 \\
      \text{in } & \textit{rewind} \; m \; (\sigma_1 \circ \sigma_2) \equiv m'' , \sigma_1' \circ \sigma_2'
    \end{array}
  \end{align*}
  and likewise for $\textit{rewind}_\subseteq$.
\item $\textit{rewind}$ should preserve identity:
  $$ \textit{rewindPresId} : (m : \Delta\lhd\Gamma) \rightarrow \textit{rewind} \; m \; id \equiv m , id $$
  and likewise for $\textit{rewind}_\subseteq$.
\item $\textit{rewind}$ should commute with weakening and substitution composition:
  \begin{align*}
    &\textit{rewindWk} &&: (m : \Delta\lhd\Gamma) \rightarrow (\sigma : \operatorname{Rpl} \Gamma \; \Gamma') \rightarrow (w : \Gamma' \subseteq \Gamma'') \\
    &&&\rightarrow
    \begin{array}[t]{@{}l@{}l@{}}
      \text{let } & m' , \sigma' = \textit{rewind} \; m \; \sigma \\
      & m'' , w' = \textit{rewind}_\subseteq \; m' \; w \\
      \text{in } & \textit{rewind} \; m \; (\textit{wk} \; w \; \sigma) \equiv m'' , \textit{wk} \; w' \; \sigma'
    \end{array} \\
    &\textit{rewindTrim} &&: (m : \Delta\lhd\Gamma) \rightarrow (w : \Gamma \subseteq \Gamma') \rightarrow (\sigma : \operatorname{Rpl} \Gamma' \; \Gamma'') \\
    &&&\rightarrow
    \begin{array}[t]{@{}l@{}l@{}}
      \text{let } & m' , w' = \textit{rewind}_\subseteq \; m \; w \\
      & m'' , \sigma' = \textit{rewind} \; m' \; \sigma \\
      \text{in } & \textit{rewind} \; m \; (\textit{trim} \; w \; \sigma) \equiv m'' , \textit{trim} \; w' \; \sigma'
    \end{array}
  \end{align*}
  where $\textit{wk} : \Delta\subseteq\Delta' \rightarrow \operatorname{Sub} \Gamma \; \Delta \rightarrow \operatorname{Sub} \Gamma \; \Delta'$ and
  $\textit{trim} : \Gamma\subseteq\Gamma' \rightarrow \operatorname{Sub} \Gamma \; \Delta \rightarrow \operatorname{Sub} \Gamma' \; \Delta$
  is substitution and weakening composition and vice versa, respectively.
\end{itemize}
These enable proving of the necessary weakening and substitution laws.
Indeed, most show up in the goals when proving the $\unbox$ cases of the laws.

\begin{theorem}[Completeness]
  If $\Gamma \vdash t : A$, then $\Gamma \vdash t \sim \ulcorner \textit{nf} \; t \urcorner$.
\end{theorem}

The proof is standard,
and established by a \emph{Kripke logical relation} \cite{kovacs17},
$t \simeq a$, between terms and semantic terms,
and $\sigma \simeq_\Gamma \hat\Gamma$, between substitutions and environments.
The most interesting case of proving the fundamental theorem of the logical relations,
i.e. $\sigma \simeq_\Gamma \delta \implies \textit{subst} \; \sigma \; t \simeq \llbracket t \rrbracket \; \delta$,
is the $\unbox$ case
where, to apply the induction hypothesis on the term to unbox,
we need $\pi_2 (\textit{rewind} \; m \; \sigma) \simeq_\Gamma \pi_2 (\textit{rewind} \; m \; \delta)$,
and a proof that $\pi_1 (\textit{rewind} \; m \; \sigma) \equiv \pi_1 (\textit{rewind} \; m \; \delta)$,
where $\pi_1$, $\pi_2$ are the first and second projections of the product type.
For now these will be most likely be calculus parameters directly.
(Although, first defining $(\simeq_\Gamma)$ as an instance of a relation
between any two $\operatorname{Rpl}$:s,
to avoid having the parameters depend on $\operatorname{Sub}$ and $\operatorname{Env}$.)
This has not yet been implemented in Agda.
It should be noted that the equivalence holds a priori,
as noted in \citetitle{wadler89} \cite{wadler89},
since a $\textit{rewind}$ instantiation cannot observe
the contents of the parametric $\operatorname{Rpl}$.
That is hard to state in Agda though, to say the least.

\section{Remaining work}

It remains to prove soundness of the conversion relation,
that is, if $\Gamma \vdash t : A$,
then $t \sim s \implies \ulcorner \textit{nf} \; t \urcorner \equiv \ulcorner \textit{nf} \; s \urcorner$.
This will be accomplished by generalizing to a presheaf model
and proceeding along the same lines as in \cite{altenkirch95}.

Another point is to investigate whether to replace OPE:s with renamings,
i.e. substitutions where the replacement terms are variables.
Not only would this remove
the $\textit{rewind}_\subseteq$/$\textit{rewind}$ repetition,
but it would also be a step toward being able to add the \emph{R axiom}
($A \rightarrow \Box A$).
\textcite{valliappan-r} showed that the condition
$$ R_m \subseteq R_i $$
on $R_i$ and $R_m$,
the intuitionistic and modal accessibility relation, respectively,
is sufficient for ensuring axiom R is satisfied in the model
(with the IR \textsc{Var} rule and corresponding OPE definition).
Unlike with OPE:s as given here,
the condition holds when picking renamings for $R_i$,
since renamings encapsulate lock weakenings.

\printbibliography

\end{document}
